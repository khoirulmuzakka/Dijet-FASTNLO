{\small \footnotesize
\begin{verbatim} 
******************************************************************************
*******                fastNLO  -  Intermediate Table for NLO  
******************************************************************************
ireaction       ! int: reaction       1 ep, 2 pp, 3ppbar
Ecms            ! dbl: center-of-mass energy in GeV
iproc           ! int: process type   1 incl jets, 2 dijets, ...
ialgo           ! int: jet algo 1 kT, 2 midpoint cone, 3 rsep cone, 4 search cone 
JetResol1       ! dbl: jet resolution paramter (kT distance D / R_cone)
JetResol2       ! dbl: 2nd parameter for jet algo (e.g. Rsep, search cone)
npow            ! int: order in alpha_s, relative:  1: LO, 2: NLO
Oalphas         ! int: order in alpha_s  absolute
1234567890      ! --- End-of-block --- now come some technical details
nevtlo          ! dbl: No of events for the LO part used to create the table
nevtnlo         ! dbl: No of events for the NLO part used to create the table
ntot            ! int: No of eigenfunctions that cover the x-ranges 
ixscheme        ! int: No of scheme for EF x-binning  1 log 1/x  2 sqrt(log 1/x)
ipdfwgt         ! int: No of scheme for PDF weighting  0 no weigthing
1234567890            ! --- End-of-block --- now come the pT,y bin boundaries
Nrapidity             ! int: No of rapidity intervals
Rap0                  ! dbl: lower boundary of 1st rapidity bin
...                   !
Rap[Nrapidity]        ! dbl: upper boundary of last rapidity bin
Npt1                  ! int: No of pT bins in first rapidity interval
...                   !
Npt[Nrapidity]        ! int: No of pT bins in last rapidity interval
Pt-1-0                ! dbl: lower boundary of 1st pT bin in 1st rapidity bin
...                   !
Pt-1-[Npt1]           ! dbl: upper boundary of last pT bin in 1st rapidity bin
...                   !
...                   !
Pt-[Nrapidity]-0      ! dbl: lower boundary of 1st pT bin in last rapidity bin
...                   ! 
Pt-[Nrapidity]-[Npt?] ! dbl: upper boundary of last pT bin in last rapidity bin
1234567890            ! --- End-of-block --- now come the x_min values for all bins
xmin-1-1              ! dbl:  xmin in 1st rapidity bin / 1st pT bin
xmin-1-2              ! dbl:  xmin in 1st rapidity bin / 2nd pT bin
...                   !
xmin-1-[Npt1]         ! dbl:  xmin in 1st rapidity bin / last pT bin
xmin-2-1              ! dbl:  xmin in 2nd rapidity bin / 1st pT bin
...                   !
xmin-[Nrap]-[Npt?]    ! dbl:  xmin in last rapidity bin / last pT bin
1234567890            ! --- End-of-block --- now come the renorm. scales
murval-1-1            ! dbl: Array for all mu_r values - for all (y,pT) bins
...                   !             (same structure as xmin-Array)
murval-[Nrap]-[Npt?]  ! dbl:
1234567890            ! --- End-of-block --- now come the factoriz. scales
mufval-1-1            ! dbl: Array for all mu_f values - for all (y,pT) bins
...                   !           (same structure as mu_r and xmin Arrays)
mufval-[Nrap]-[Npt?]  ! dbl:
1234567890            ! --- End-of-block --- now come the scale variations
nscalevar             ! int: No of scale variations for mu_r and mu_f, this *includes* the standard scale
murscale[nscalevar]   ! dbl: renormalisation scale factors with respect to standard scale SQUARED
mufscale[nscalevar]   ! dbl: factorisation scale factors with respect to standard scale SQUARED
1234567890            ! --- End-of-block --- now come the sigma_tilde
...                   ! dbl: each rapidity range
...                   !        each pT bin
...                   !           nscalevar variations:
...                   !             1:NLO, standard scale, 2:NLO, 1st variation
...                   !             3:NLO, 2nd variation, ..... ]
...                   !            seven subprocesses:     
...                   !            [each (of five) subprocesses:
...                   !             half-grid with (n**2+n)/2  x-bins
...                   !             remaining two subprocesses: 
...                   !             can be combined to one full n**2 grid
...                   !             (or better: two half grids)]
1234567890            ! --- End of Table ------------------------------------
\end{verbatim}
}
