{\small \footnotesize
\begin{verbatim} 
******************************************************************************
The table contains the following blocks (separated by ``1234567890''):
- version No and date
- scenario description
- observable-specific flags (reaction, ECM, jet algo, order, available orders...)
- details about the QCD calculation
- technical fastNLO details (Nevt, Nxtot, Ixscheme, Ipdfwgt, ...)
- structure of observable: tot No. of bins, and structure of dimensions in binning
- bin boundaries 
- xmin values
- renormalization scales
- factorization scales
- scale variations
- the sigma-tilde
******************************************************************************
*******                fastNLO  -  general definition of tableformat
*******                            for raw (Nord=1,etc) and sum tables
******************************************************************************
VersionNo          ! dbl: Version of Table - for compatibility check (e.g.1.01)
idate              ! int: date when sum table was created (yyyymmdd)
* ---- these first two entries are added to the table by nlofast-add !!!!! <<<<<<<
*      they are not contained in the raw tables

1234567890         ! --- End-of-block --- now the scenario description
ScenName           ! string:  name of scenario  (e.g. "fnt1001")
iSDescr            ! int:     No. of Strings for Description of scenario
ScenDescr[1]       ! string:  Description of scenario
...                !      (e.g. hep-ex No., Collaboration, ...)
ScenDescr[iSDescr] ! 

1234567890      ! --- End-of-block --- now some observable-specific flags
ireaction       ! int: reaction: 1 ep, 2 pp, 3ppbar,4gamma-p(dir), 5gamma-p(resolved)
Ecms            ! dbl: center-of-mass energy in GeV
iproc           ! int: process type   1 incl jets, 2 dijets, ... (just a little info...)
ialgo           ! int: jet algo 1 kT, 2 midpoint cone, 3 rsep cone, 4 search cone 
JetResol1       ! dbl: jet resolution paramter (kT distance D / R_cone)
JetResol2       ! dbl: 2nd parameter for jet algo (e.g. f_overlap, Rsep, search cone)

1234567890          ! --- End-of-block --- now some info on the QCD calculation
iasflg              ! int:  Flag (0: w/o alphas   1: alphas already contained in table)
asMZ                ! dble: for iasflg=1:  asMZ is alphas(Mz)
ipdfflg             ! int:  Flag (0: w/o PDFs     1: PDFs already included in table)
PDFset              ! string: for ipdfflg=1:  PDF set included in table (eg:"CTEQ6.1")
>> for resolved Photoproduction only:
  igampdfflg        ! int:  Flag (0: w/o photonPDFs   1: photon-PDFs already included in table)
  gamPDFset         ! string: for igampdfflg=1:  photon-PDF set included in table
nFlav               ! int: No of active quark flavors used in calculation
Nord                ! int:  * in raw table: No. of this table in sum table
                    !                          / usually = relative order 
                    !                          in alpha_s:  1: LO, 2: NLO, 
                    !                          3:NNLO or other correction
                    !       * in sum table:  tot. No. of single tables in sum table 
                    !                        in fixed-order calculations =
                    !                        highest relative order available
* in raw table:
Npow                ! int: absolute order in alpha_s of this raw table
* in sum table:
Npow1               ! int: absolute order in alpha_s for each contribution [1,...,Nord]
...                 !      (to be used in usercode as power of alpha_s) 
Npow[Nord]
* in raw table:
Powlabel        ! string: label for the order, e.g ``LO''
* in sum table:
Powlabel1       ! string: label for the order, for each contribution [1,...,Nord]
...                
Powlabel[Nord]
ScaleA        ! string: default scale  (e.g. "pT", "ETmax", "Q", "Mjj"       <<<<<<<<<<<<<<<<< new
ScaleB        ! string: short description (e.g. computed from two highest pT jets) <<<<<<<<<<< new
ScaleC        ! string: more description (e.g. "taken at 40% of the bin")    <<<<<<<<<<<<<<<<< new

1234567890      ! --- End-of-block --- now come some technical fastNLO details
* in raw table:
Nevt            ! dbl: No of events for this raw table
* in sum table:
Nevt1           ! dbl: No of events for contribution 1 (usually LO)
Nevt2           ! dbl: No of events for contribution 2 (usually NLO)
...             ! dbl: No of events for higher order contributions 
Nevt[Nord]      !                   (or resummed correction)
Nxtot           ! int: No of eigenfunctions in that cover the x-ranges 

ixscheme        ! int: No of scheme for EF x-binning  1 log 1/x  2 sqrt(log 1/x)
ipdfwgt         ! int: No of scheme for PDF weighting  0 no weigthing  1 'standard weighting'
>> for resolved Photoproduction only:
  igampdfwgt    ! int: No of scheme for photon-PDF weighting  0 no weigthing  1 'standard weighting'

*** previously we had the No of rap bins together with the rapidity boundaries
*** -> this is now separated: first come all array sizes - afterwards we
***    declare all values (upper & lower values for each interval)

1234567890            ! --- End-of-block --- now: structure of the dimensions
Nbintot               ! int: total Number of observable bins
Ndimension            ! int: No. of dimensions in variable is binned
                      !    3-d example: Jet Shape Psi as function of (y,pT,r)
Cdim[1]               ! string: which variable corresponds to 1st dimension (eg: ``y'')
...                   !
Cdim[Ndimension]      ! string: which variable corresponds to lasst dimension
                      !
Ndim1                 ! int: No. of intervals in first dimension 
                      !           (e.g.: No. of y-ranges)

Ndim2[1]              ! int: No. of intervals in 2nd dim for each interval in 1st dim 
....                  !           (e.g.: No. of pT ranges in each y-range)
Ndim2[Ndim1]          !    


Ndim3-[1]-[1]            ! int: No. of intervals in 3rd dim for each interval in 2nd dim 
...                      !       - for each interval in 1st dim 
Ndim3-[1]-[Ndim2[1]]     !    (e.g.: No. of r bins in each pT range - in each y-range)
...                       !
...                         !
Ndim3-[Ndim1]-[1]            !
...                          !
Ndim3-[Ndim1]-[Ndim2[Ndim1]] !


Ndim'Ndimension'- ..-..-..-..-..

   *********** I don't want to write out more than the third dimension   :-)

1234567890     ! --- End-of-block --- now come the bin boundaries - for all dimensions

************* not sure: should we first fill all lower values - then all higher values?
************* or: always pairs low/high?   I choose the second one for now
Dim1lo[1]      ! dbl:    lower and higher boundary for 1st interval in dim 1
Dim1hi[1]      !
...            !
Dim1lo[Ndim1]  !         lower and higher boundary for last interval in dim 1
Dim1hi[Ndim1]  ! 

Dim2lo[1]-[1]        ! dbl:  lower and higher boundary for 1st interval in dim1/dim2
Dim2hi[1]-[1]        !
...
Dim2lo[1]-[Ndim2[1]] ! dbl:  lower and higher boundary for 1st interval in dim 1
Dim2hi[1]-[Ndim2[1]] !
...
...
...           
Dim2lo[Ndim1]-[1] 
Dim2hi[Ndim1]-[1] 
...
Dim2lo[Ndim1]-[Ndim2[Ndim1]]
Dim2hi[Ndim1]-[Ndim2[Ndim1]]

     *** 3rd dimension - if available ...

Dim3lo[1]-[1]-[1]      ! dbl:    lower and higher boundary for 1st interval in dim1/dim2
Dim3hi[1]-[1]-[1]      !
...
Dim3lo[1]-[1]-[Ndim3[1]-[1]]
Dim3hi[1]-[1]-[Ndim3[1]-[1]] 
...
...
...
Dim3lo[1]-[Ndim2[1]-[1]     
Dim3hi[1]-[Ndim2[1]-[1]     
...
Dim3lo[1]-[Ndim2[1]-[Ndim3[1]-[[1]]
Dim3hi[1]-[Ndim2[1]-[Ndim3[1]-[[1]] 
....
       ************ I'm getting lost here ......

1234567890    ! --- End-of-block --- now come the x_min values for all bins
xmin-1-1-?    ! dbl:  xmin in 1st rapidity bin / 1st pT bin / 1st r bin
....          !          and all other xmin-values 
              !          - same structure as bin-boundaries
>> for resolved Photoproduction only:
  xgammamin-1-1-?    ! dbl:  xmin in 1st rapidity bin / 1st pT bin / 1st r bin
  ....          !          and all other xmin-values 
                !          - same structure as bin-boundaries

1234567890    ! --- End-of-block --- now come the renorm. scales for all bins
murval-1-1-?  ! dbl: Array for all mu_r values - 
...           !             (same structure as bin-boundaries)

1234567890    ! --- End-of-block --- now come the factoriz. scales for all bins
mufval-1-1-?  ! dbl: Array for all mu_f values - for all (y,pT) bins
  ...           !             (same structure as bin-boundaries)
>> for resolved Photoproduction only:
  mufvalgamma-1-1-?  ! dbl: Array for all mu_f-gamma values - for all (y,pT) bins
  ...                !             (same structure as bin-boundaries)


* - the following is not used in LO tables - only in higher order raw tables and in sum tables
1234567890            ! --- End-of-block --- now come the scale variations
nscalevar             ! int: No of scale variations for mu_r and mu_f
murscale[1]           ! dbl: renormalisation scale factors with respect to standard scale 
...                   !
murscale[nscalevar]   !
mufscale[1]           ! dbl: factorisation scale factors with respect to standard scale 
...                   !
mufscale[nscalevar]   ! 
>> for resolved Photoproduction only:
  mufscalegamma[1]         ! dbl: factorisation scale factors with respect to standard scale 
  ...                      !
  mufscalegamma[nscalevar] ! 

1234567890            ! --- End-of-block --- now come the sigma_tilde
...                   ! dbl: each range in dimension 1
...                   !        each range in dimension 2
...                   !          ... each range in dimension n
...                   !            loop over xmax bins
...                   !              loop over xmin bins (not for ep)
...                   !                     -> for pp from xmin->xmax             
...                   !                     -> for photoproduction from xmin->1  
...                   !                loop over scale variations (not for LO table, i.e. Nord=1)
...                   !                  loop over subprocesses (ep:3, pp:7)

???????????? there is a difference in authorcode and nlofast-add:
???????????? authorcode first loops over scales - then subprocesses (as above)
???????????? nlofast-add first loops over subprocesses (when writing!) - then scales
???????????? --> should be the same!!!

1234567890            ! --- End of Table ------------------------------------
\end{verbatim}
}


