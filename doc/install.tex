{\small \footnotesize
\begin{verbatim} 
******************************************************************************
*******          how to install and run fastNLO 
******************************************************************************

* ---------------------------------------------------------
* step 1: installing a slightly modified NLOJET++ version
* ---------------------------------------------------------
To use all features of the new fastNLO code you need to
install a slightly modified version of NLOJET++
(includes only one small modification - see below):
* create a new directory: NLOJET
* get the code from Zoltan Nagy's webpage  
  http://www.cpt.dur.ac.uk/~nagyz/nlo++/
  (so far fastNLO is using NLOJET++ version 2.0.1)
* copy the file nlojet++-2.0.1.tar.gz into the new directory NLOJET 
  and unpack the code using:
       tar -xzf nlojet++-2.0.1.tar.gz
* 2 minor modifications of the NLOJET++ code are now needed:
  - bugfix for DIS code
    in the file:     include/bits/nlo-process_dis.h
    change line 332 to:   for(unsigned int i = 0; i < 5; i++)
  - patch needed for fastNLO
     change line 40 in the file:  include/bits/nlo-basic_user.h
     from:        void phys_output(const std::basic_..... 
     to:          virtual void phys_output(const std::basic_..... 
* compile NLOJET++ (may take 15 minutes) using:
    1) cd nlojet++-2.0.1
    2) aclocal
    3) automake
    4) autoconf
    5) ./configure --prefix=[complete path of your NLOJET directory]
    6) make
    7) make install

 .... or here are Zoltan's original instructions:
       mkdir bld-nlojet 
       cd bld-nlojet
       ../nlojet++-2.0.1/configure --prefix [complete path of your NLOJET directory]
       make install CFLAGS="-O3 -Wall" CXXFLAGS="-O3 -Wall"


* ---------------------------------------------------------
* step 2: installing and running the fastNLO author code
* ---------------------------------------------------------

* create a new subdirectory "fastNLO" in your NLOJET directory:
       mkdir fastNLO
* copy the authorcode code from SVN into the fastNLO directory
* compile the fstNLO author code
        make [.....]
* then create a directory or a link "/path" to where your tables 
  should be stored
* then run:
../bin/nlojet++ -A dipole --save-after 50000000 -cborn -n run1incl01a00 -d /path -u fnrun1incl01a.la
../bin/nlojet++ -A dipole --save-after 5000000 -cnlo -n run1incl01a00 -d /path -u fnrun1incl01a.la

(stores tables under /path)

* comments: 
  - please check the number of events after  --save-after  
    in a testjob to make sure that you don't store your
    tables too frequent or to infreqent. 1-2h makes sense

* ---------------------------------------------------------
* step 3: create fastNLO sum tables
* ---------------------------------------------------------
in directory /path: use nlofast-add

./nlofast-add run1incl01a00-hhc-born-2jet.raw run1incl01a01-hhc-nlo-2jet.raw run1incl01.tab

* ---------------------------------------------------------
* step 4: run the fastNLO user code
* ---------------------------------------------------------
run user code (in SVN directory /user1c )
set paths for CERNLIB and LHAPDF (v4.2) in makefile (or run source fastNLO.csh [user])
in example01.f: set path to CTEQ6.1M grid in LHAPDF directory:
    call InitPDFset('/disk2/work/wobisch/lhapdf-4.2/PDFsets/cteq61.LHgrid')


make -> then run: 
./example /path/mytable.tab myhbookfile.hbk


* ---------------------------------------------------------
* step 5: plot histograms
* ---------------------------------------------------------
(example kumac is not up to date)

\end{verbatim}
}


